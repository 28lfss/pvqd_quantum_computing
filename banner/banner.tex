\documentclass{article}

\usepackage[
  paperwidth=90cm,
  paperheight=120cm,
  left=2cm,
  right=2cm,
  top=2cm,
  bottom=2cm
]{geometry}

\usepackage{graphicx}
\usepackage[most]{tcolorbox}
\usepackage{xcolor}
\usepackage{titlesec}
\usepackage{helvet}
\usepackage{setspace}
\usepackage{booktabs}
\usepackage{array}
\usepackage{colortbl}
\usepackage{listings}
\usepackage{amsmath}
\usepackage{tikz}

% Cores acadêmicas profissionais
\definecolor{academicblue}{RGB}{0,51,102}
\definecolor{lightblue}{RGB}{230,240,250}
\definecolor{mediumgray}{RGB}{128,128,128}
\definecolor{sectionblue}{RGB}{0,76,153}

\renewcommand{\familydefault}{\sfdefault}

\renewcommand\normalsize{\fontsize{30pt}{36pt}\selectfont}
\newcommand{\TitleFont}{\fontsize{52pt}{62pt}\selectfont}
\newcommand{\AuthorFont}{\fontsize{40pt}{48pt}\selectfont}
\newcommand{\RefFont}{\fontsize{20pt}{24pt}\selectfont}

\onehalfspacing

\titleformat{\section}
  {\bfseries\Large\color{sectionblue}\uppercase}
  {}
  {0pt}
  {\rule[-0.3cm]{0.3cm}{0.3cm}\hspace{0.5cm}}

\lstdefinestyle{pythonbanner}{
  language=Python,
  basicstyle=\fontsize{24pt}{28pt}\ttfamily\color{academicblue},
  keywordstyle=\bfseries\color{sectionblue},
  commentstyle=\itshape\color{mediumgray},
  stringstyle=\color{academicblue},
  showstringspaces=false,
  frame=leftline,
  framerule=3pt,
  rulecolor=\color{academicblue},
  breaklines=true,
  backgroundcolor=\color{lightblue}
}

% Estilo para caixas principais
\tcbset{
  colframe=academicblue,
  colback=lightblue,
  boxrule=1.5pt,
  arc=8pt,
  left=10pt,
  right=10pt,
  top=10pt,
  bottom=10pt,
  fonttitle=\bfseries,
  titlerule=0.5pt,
  title style={fill=academicblue, text=white}
}

\begin{document}
\thispagestyle{empty}

% ======================================================
% CABEÇALHO
% ======================================================

\begin{center}
{\TitleFont \textcolor{academicblue}{\textbf{PROJEÇÃO VARIACIONAL DA DINÂMICA QUÂNTICA (PVQD)}}\\}

\vspace{0.8cm}

{\AuthorFont \textcolor{mediumgray}{Autores: Lucas Fernandes Brayner; Luiz Felipe Sá; Victor Silva Dutra}}

\vspace{0.3cm}

\normalsize \textcolor{mediumgray}{Turma: Ciências da Computação 8NA / Instituição: Uninassau}\\
\textcolor{mediumgray}{Orientador(a): Antenor Jorge Parnaíba}
\end{center}

\vspace{0.3cm}
\begin{center}
\color{academicblue}\rule{0.8\linewidth}{2.5pt}
\end{center}
\vspace{0.8cm}

% ======================================================
% DUAS COLUNAS MANUAIS COM MINIPAGE
% ======================================================

\begin{minipage}[t]{0.48\linewidth}

% ---------- INTRODUÇÃO ----------
\section*{Introdução}

\begin{tcolorbox}
A Projeção Variacional da Dinâmica Quântica (PVQD) é um método híbrido
quântico–clássico que aproxima a evolução temporal de um sistema quântico
por meio de um \emph{ansatz} parametrizado, ajustado passo a passo a partir
da equação de Schr\"odinger dependente do tempo.
\end{tcolorbox}

\begin{center}
\fcolorbox{academicblue}{white}{\includegraphics[width=0.95\linewidth]{images/pvqd_geometry_flow.png}}\\[0.2cm]
{\RefFont \textcolor{mediumgray}{Esquema conceitual do método PVQD.}}
\end{center}

\vspace{0.8cm}

% ---------- FUNDAMENTAÇÃO TEÓRICA ----------
\section*{Fundamentação teórica}

\begin{tcolorbox}
A dinâmica de um sistema quântico fechado é descrita pela equação de
Schr\"odinger dependente do tempo:
\[
  i \frac{d}{dt}\lvert \psi(t)\rangle = H \lvert \psi(t)\rangle .
\]

\end{tcolorbox}

\vspace{0.4cm}

\begin{tcolorbox}
No PVQD, o estado é restringido a uma família variacional
$\lvert \psi(\boldsymbol{\theta}(t))\rangle$ gerada por um circuito
\emph{ansatz}. A evolução exata é projetada nesse espaço por meio da
minimização da norma:
\[
  \Big\|
    \tfrac{d}{dt} \lvert \psi(\boldsymbol{\theta})\rangle
    - H \lvert \psi(\boldsymbol{\theta})\rangle
  \Big\|^2 ,
\]
levando a um sistema linear para $\dot{\boldsymbol{\theta}}$ em cada passo
de tempo.
\end{tcolorbox}

\vspace{0.8cm}

% ---------- OBJETIVOS ----------
\section*{Objetivos}

\begin{tcolorbox}
\textbf{Objetivo geral}\\[0.2cm]
Aplicar o método PVQD ao modelo de Ising em 2 qubits, avaliando sua
capacidade de reproduzir a evolução temporal com alta fidelidade.\\[0.4cm]
\textbf{Objetivos específicos}
\begin{itemize}
  \item Construir o Hamiltoniano de Ising com acoplamento $Z Z$ e
        campo transversal $X$.
  \item Implementar um ansatz parametrizado adequado para 2 qubits.
  \item Simular a evolução variacional usando Qiskit em modo
        \emph{statevector}.
  \item Calcular fidelidade, energia $\langle H\rangle$ e correlação
        $\langle Z Z\rangle$ ao longo do tempo.
\end{itemize}
\end{tcolorbox}

\vspace{0.6cm}

% ---------- METODOLOGIA ----------
\section*{Metodologia}

\begin{tcolorbox}
O Hamiltoniano de Ising inclui acoplamentos $Z Z$ e campo transversal $X$,
enquanto a dinâmica é aproximada por um circuito ansatz com rotações locais
e portas de interação. O fluxo do algoritmo PVQD pode ser resumido em:
\begin{enumerate}
  \item Definir $H$ e o ansatz $\lvert \psi(\boldsymbol{\theta})\rangle$.
  \item Calcular as matrizes variacionais e o vetor de forças.
  \item Resolver o sistema linear para $\dot{\boldsymbol{\theta}}$.
  \item Atualizar os parâmetros e avançar no tempo.
  \item Medir observáveis de interesse em cada passo.
\end{enumerate}
\end{tcolorbox}

\vspace{0.4cm}

\begin{center}
\fcolorbox{academicblue}{white}{\includegraphics[width=0.75\linewidth]{images/ansatz_layer.png}}\\[0.2cm]
{\RefFont \textcolor{mediumgray}{Camada do ansatz com rotações $R_X$ e acoplamento $Z Z(\theta)$.}}
\end{center}

\end{minipage}
\hfill
\begin{minipage}[t]{0.48\linewidth}

% ---------- CÓDIGO ----------
\section*{Código utilizado}

\begin{tcolorbox}
\begin{lstlisting}[style=pythonbanner]
from config import PVQDConfig
from hamiltonian import create_ising_hamiltonian
from circuit import create_ansatz
from pvqd_solver import PVQDSolver
from results import ResultsProcessor

def main():
    config = PVQDConfig()
    h = create_ising_hamiltonian(config)
    ans = create_ansatz(config)
    solver = PVQDSolver(config)

    result = solver.solve(
        hamiltonian=h,
        aux_operators=[h]
    )

    ResultsProcessor(
        solver.estimator
    ).display_results(result, h)
\end{lstlisting}
\end{tcolorbox}

\vspace{0.8cm}

% ---------- RESULTADOS ----------
\section*{Resultados}

\begin{tcolorbox}
\textbf{Parâmetros da simulação}
\begin{itemize}
  \item Qubits: 2
  \item Passos de tempo: 10
  \item Tempo total: $T = 1{,}0$
  \item Parâmetros do ansatz: 8
  \item Otimizador clássico com até 20 avaliações por passo.
\end{itemize}
\end{tcolorbox}

\vspace{0.4cm}

\begin{tcolorbox}
\textbf{Fidelidade}\\[0.2cm]
A fidelidade entre o estado variacional e o estado de referência exato
permaneceu elevada ao longo de toda a evolução:
\begin{center}
\arrayrulecolor{academicblue}
\begin{tabular}{>{\color{academicblue}}c >{\color{academicblue}}c}
\toprule[1.5pt]
$t$ & Fidelidade \\
\midrule[0.8pt]
0.0 & 1.0000 \\
0.5 & 0.9990 \\
1.0 & 0.9980 \\
\bottomrule[1.5pt]
\end{tabular}
\end{center}
O valor mínimo observado foi $\mathcal{F}_{\min} \approx 0{,}998$.
\end{tcolorbox}

\vspace{0.4cm}

\begin{tcolorbox}
\textbf{Observáveis energéticos e de correlação}\\[0.2cm]
\begin{center}
\arrayrulecolor{academicblue}
\begin{tabular}{>{\color{academicblue}}c >{\color{academicblue}}c >{\color{academicblue}}c}
\toprule[1.5pt]
$t$ & $\langle H\rangle$ & $\langle Z Z\rangle$ \\
\midrule[0.8pt]
0.0 & 0.500000 & 1.000000 \\
0.5 & 0.503872 & 0.999598 \\
1.0 & 0.507375 & 0.998441 \\
\bottomrule[1.5pt]
\end{tabular}
\end{center}
A energia varia suavemente com o tempo, enquanto a correlação
$\langle Z Z\rangle$ permanece próxima de 1, indicando fortes
correlações de spin entre os qubits.
\end{tcolorbox}

\vspace{0.6cm}

% ---------- DISCUSSÃO ----------
\section*{Discussão}

\begin{tcolorbox}
Os resultados indicam que um ansatz relativamente raso, com apenas 8
parâmetros, é suficiente para capturar a dinâmica do modelo de Ising em
2 qubits com alta fidelidade. O PVQD se mostra uma alternativa promissora a esquemas baseados em
Trotterização, pois substitui decomposições longas de $e^{-iHt}$ por
um problema de otimização com menor profundidade e número de portas
entrelançadas.
\end{tcolorbox}

\vspace{0.6cm}

% ---------- REFERÊNCIAS ----------
\section*{Referências}

\begin{tcolorbox}
\RefFont
1. Qiskit Algorithms. \emph{Projected Variational Quantum Dynamics Tutorial}.\\
   Disponível em: \texttt{https://qiskit-community.github.io/qiskit-algorithms/tutorials/10\_pvqd.html}\\[0.1cm]
2. BARISON, S.; VICENTINI, F.; CARLEO, G. An efficient quantum algorithm
   for the time evolution of parameterized circuits. \emph{Quantum}, v. 5, p. 512, 2021.\\
   Disponível em: \texttt{https://quantum-journal.org/papers/q-2021-07-28-512/}\\[0.1cm]
3. IBM Quantum. \emph{PVQD API Documentation}.\\
   Disponível em: \texttt{https://quantum.cloud.ibm.com/docs/en/api/qiskit/0.46/qiskit.algorithms.PVQD}\\[0.1cm]
4. BARISON, S.; VICENTINI, F.; CARLEO, G. An efficient quantum algorithm
   for the time evolution of parameterized circuits. arXiv:2101.04579, 2021.\\
   Disponível em: \texttt{https://arxiv.org/abs/2101.04579}
\end{tcolorbox}

\end{minipage}

\end{document}
