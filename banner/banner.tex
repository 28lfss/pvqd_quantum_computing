\documentclass{article}

\usepackage[
  paperwidth=90cm,
  paperheight=120cm,
  left=2cm,
  right=2cm,
  top=2cm,
  bottom=2cm
]{geometry}

\usepackage{graphicx}
\usepackage[most]{tcolorbox}
\usepackage{xcolor}
\usepackage{titlesec}
\usepackage{helvet}
\usepackage{setspace}
\usepackage{booktabs}
\usepackage{array}
\usepackage{listings}
\usepackage{amsmath}

\renewcommand{\familydefault}{\sfdefault}

\renewcommand\normalsize{\fontsize{30pt}{36pt}\selectfont}
\newcommand{\TitleFont}{\fontsize{52pt}{62pt}\selectfont}
\newcommand{\AuthorFont}{\fontsize{40pt}{48pt}\selectfont}
\newcommand{\RefFont}{\fontsize{20pt}{24pt}\selectfont}

\onehalfspacing

\titleformat{\section}
  {\bfseries\uppercase}
  {}
  {0pt}
  {}

\lstdefinestyle{pythonbanner}{
  language=Python,
  basicstyle=\fontsize{22pt}{26pt}\ttfamily,
  keywordstyle=\bfseries,
  showstringspaces=false,
  frame=single,
  breaklines=true
}

\tcbset{
  colframe=black,
  colback=white,
  boxrule=1pt,
  arc=5pt,
  left=6pt,
  right=6pt,
  top=6pt,
  bottom=6pt
}

\begin{document}
\thispagestyle{empty}

% ======================================================
% CABEÇALHO
% ======================================================

\begin{center}
{\TitleFont \textbf{PROJEÇÃO VARIACIONAL DA DINÂMICA QUÂNTICA (PVQD)}\\}

\vspace{0.5cm}

{\AuthorFont Autores: Lucas Fernandes Brayner; Luiz Felipe Sá; Victor Silva Dutra}

\normalsize Ciências da Computação / Uninassau\\
Orientador(a): Antenor Jorge Parnaíba
\end{center}

\vspace{1cm}
\hrule
\vspace{1cm}

% ======================================================
% DUAS COLUNAS MANUAIS COM MINIPAGE
% ======================================================

\begin{minipage}[t]{0.48\linewidth}

% ---------- INTRODUÇÃO ----------
\section*{Introdução}

\begin{tcolorbox}
A Projeção Variacional da Dinâmica Quântica (PVQD) é um método híbrido
quântico–clássico que aproxima a evolução temporal de um sistema quântico
por meio de um \emph{ansatz} parametrizado, ajustado passo a passo a partir
da equação de Schr\"odinger dependente do tempo.
\end{tcolorbox}

\begin{center}
\includegraphics[width=0.95\linewidth]{images/pvqd_geometry_flow.png}\\
{\RefFont Esquema conceitual do método PVQD.}
\end{center}

\vspace{0.8cm}

% ---------- FUNDAMENTAÇÃO TEÓRICA ----------
\section*{Fundamentação teórica}

\begin{tcolorbox}
A dinâmica de um sistema quântico fechado é descrita pela equação de
Schr\"odinger dependente do tempo:
\[
  i \frac{d}{dt}\lvert \psi(t)\rangle = H \lvert \psi(t)\rangle .
\]

\end{tcolorbox}

\vspace{0.4cm}

\begin{tcolorbox}
No PVQD, o estado é restringido a uma família variacional
$\lvert \psi(\boldsymbol{\theta}(t))\rangle$ gerada por um circuito
\emph{ansatz}. A evolução exata é projetada nesse espaço por meio da
minimização da norma:
\[
  \Big\|
    \tfrac{d}{dt} \lvert \psi(\boldsymbol{\theta})\rangle
    - H \lvert \psi(\boldsymbol{\theta})\rangle
  \Big\|^2 ,
\]
levando a um sistema linear para $\dot{\boldsymbol{\theta}}$ em cada passo
de tempo.
\end{tcolorbox}

\vspace{0.8cm}

% ---------- OBJETIVOS ----------
\section*{Objetivos}

\begin{tcolorbox}
\textbf{Objetivo geral}\\[0.2cm]
Aplicar o método PVQD ao modelo de Ising em 2 qubits, avaliando sua
capacidade de reproduzir a evolução temporal com alta fidelidade.\\[0.4cm]
\textbf{Objetivos específicos}
\begin{itemize}
  \item Construir o Hamiltoniano de Ising com acoplamento $Z Z$ e
        campo transversal $X$.
  \item Implementar um ansatz parametrizado adequado para 2 qubits.
  \item Simular a evolução variacional usando Qiskit em modo
        \emph{statevector}.
  \item Calcular fidelidade, energia $\langle H\rangle$ e correlação
        $\langle Z Z\rangle$ ao longo do tempo.
\end{itemize}
\end{tcolorbox}

\vspace{0.6cm}

% ---------- METODOLOGIA ----------
\section*{Metodologia}

\begin{tcolorbox}
O Hamiltoniano de Ising inclui acoplamentos $Z Z$ e campo transversal $X$,
enquanto a dinâmica é aproximada por um circuito ansatz com rotações locais
e portas de interação. O fluxo do algoritmo PVQD pode ser resumido em:
\begin{enumerate}
  \item Definir $H$ e o ansatz $\lvert \psi(\boldsymbol{\theta})\rangle$.
  \item Calcular as matrizes variacionais e o vetor de forças.
  \item Resolver o sistema linear para $\dot{\boldsymbol{\theta}}$.
  \item Atualizar os parâmetros e avançar no tempo.
  \item Medir observáveis de interesse em cada passo.
\end{enumerate}
\end{tcolorbox}

\vspace{0.4cm}

\begin{center}
\includegraphics[width=0.75\linewidth]{images/ansatz_layer.png}\\
{\RefFont Camada do ansatz com rotações $R_X$ e acoplamento $Z Z(\theta)$.}
\end{center}

\end{minipage}
\hfill
\begin{minipage}[t]{0.48\linewidth}

% ---------- CÓDIGO ----------
\section*{Código utilizado}

\begin{tcolorbox}
\begin{lstlisting}[style=pythonbanner]
from config import PVQDConfig
from hamiltonian import create_ising_hamiltonian
from circuit import create_ansatz
from pvqd_solver import PVQDSolver
from results import ResultsProcessor

def main():
    config = PVQDConfig()
    h = create_ising_hamiltonian(config)
    ans = create_ansatz(config)
    solver = PVQDSolver(config)

    result = solver.solve(
        hamiltonian=h,
        aux_operators=[h]
    )

    ResultsProcessor(
        solver.estimator
    ).display_results(result, h)
\end{lstlisting}
\end{tcolorbox}

\vspace{0.8cm}

% ---------- RESULTADOS ----------
\section*{Resultados}

\begin{tcolorbox}
\textbf{Parâmetros da simulação}
\begin{itemize}
  \item Qubits: 2
  \item Passos de tempo: 10
  \item Tempo total: $T = 1{,}0$
  \item Parâmetros do ansatz: 8
  \item Otimizador clássico com até 20 avaliações por passo.
\end{itemize}
\end{tcolorbox}

\vspace{0.4cm}

\begin{tcolorbox}
\textbf{Fidelidade}\\[0.2cm]
A fidelidade entre o estado variacional e o estado de referência exato
permaneceu elevada ao longo de toda a evolução:
\begin{center}
\begin{tabular}{cc}
\toprule
$t$ & Fidelidade \\
\midrule
0.0 & 1.0000 \\
0.5 & 0.9990 \\
1.0 & 0.9980 \\
\bottomrule
\end{tabular}
\end{center}
O valor mínimo observado foi $\mathcal{F}_{\min} \approx 0{,}998$.
\end{tcolorbox}

\vspace{0.4cm}

\begin{tcolorbox}
\textbf{Observáveis energéticos e de correlação}\\[0.2cm]
\begin{center}
\begin{tabular}{ccc}
\toprule
$t$ & $\langle H\rangle$ & $\langle Z Z\rangle$ \\
\midrule
0.0 & 0.500000 & 1.000000 \\
0.5 & 0.503872 & 0.999598 \\
1.0 & 0.507375 & 0.998441 \\
\bottomrule
\end{tabular}
\end{center}
A energia varia suavemente com o tempo, enquanto a correlação
$\langle Z Z\rangle$ permanece próxima de 1, indicando fortes
correlações de spin entre os qubits.
\end{tcolorbox}

\vspace{0.6cm}

% ---------- DISCUSSÃO ----------
\section*{Discussão}

\begin{tcolorbox}
Os resultados indicam que um ansatz relativamente raso, com apenas 8
parâmetros, é suficiente para capturar a dinâmica do modelo de Ising em
2 qubits com alta fidelidade. O PVQD se mostra uma alternativa promissora a esquemas baseados em
Trotterização, pois substitui decomposições longas de $e^{-iHt}$ por
um problema de otimização com menor profundidade e número de portas
entrelançadas.
\end{tcolorbox}

\vspace{0.6cm}

% ---------- REFERÊNCIAS ----------
\section*{Referências essenciais}

\begin{tcolorbox}
\RefFont
1. A. O. Author \emph{et al.}, ``Projected Variational Quantum Dynamics'',
   Journal X, ano.\\[0.1cm]
2. M. Nielsen, I. Chuang, \emph{Quantum Computation and Quantum Information}.\\[0.1cm]
3. Documentação Qiskit: \texttt{https://qiskit.org}
\end{tcolorbox}

\end{minipage}

\end{document}
