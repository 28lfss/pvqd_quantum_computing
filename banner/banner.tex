\documentclass{article}

% ----------------- TAMANHO DO BANNER -----------------
% 0,90 m de largura x 1,20 m de altura (requisitos FEBIC)
\usepackage[
  paperwidth=90cm,
  paperheight=120cm,
  left=2cm,
  right=2cm,
  top=2cm,
  bottom=2cm
]{geometry}

% ----------------- PACOTES GRÁFICOS ------------------
\usepackage{graphicx}
\usepackage[most]{tcolorbox}
\usepackage{multicol}
\usepackage{xcolor}
\usepackage{titlesec}
\usepackage{helvet}
\usepackage{setspace}
\usepackage{booktabs}
\usepackage{array}
\usepackage{listings}

% Fonte sem serifa (estilo de banner)
\renewcommand{\familydefault}{\sfdefault}

% ----------------- TAMANHOS DE FONTE -----------------
% Texto principal ~30 pt (como sugerido nas orientações)
\renewcommand\normalsize{\fontsize{30pt}{36pt}\selectfont}
\newcommand{\TitleFont}{\fontsize{60pt}{72pt}\selectfont}
\newcommand{\AuthorFont}{\fontsize{40pt}{48pt}\selectfont}
\newcommand{\RefFont}{\fontsize{20pt}{24pt}\selectfont}

% Espaçamento de linhas
\onehalfspacing

% ----------------- ESTILO DE SEÇÃO --------------------
\titleformat{\section}
  {\bfseries\uppercase}
  {}
  {0pt}
  {}

% ----------------- CONFIGURAÇÃO DO LISTINGS (CÓDIGO) --
\lstdefinestyle{pythonbanner}{
  language=Python,
  basicstyle=\fontsize{22pt}{26pt}\ttfamily,
  keywordstyle=\bfseries,
  showstringspaces=false,
  frame=single,
  breaklines=true
}

% ----------------- CONFIGURAÇÃO DAS CAIXAS ------------
\tcbset{
  colframe=black,
  colback=white,
  boxrule=1pt,
  arc=5pt,
  left=6pt,
  right=6pt,
  top=6pt,
  bottom=6pt
}

% ------------------------------------------------------
\begin{document}
\thispagestyle{empty}

% ======================================================
%             FAIXA SUPERIOR: LOGOS + TÍTULO
% ======================================================
\noindent
\begin{minipage}[c]{0.18\linewidth}
  \centering
  \includegraphics[width=\linewidth]{images/uninassau.png}\\[0.5cm]
\end{minipage}
\begin{minipage}[c]{0.64\linewidth}
  \centering
  {\TitleFont\textbf{PROJEÇÃO VARIACIONAL DA DINÂMICA QUÂNTICA (PVQD)\\
  APLICADA AO MODELO DE ISING EM 2 QUBITS}}\\[0.8cm]
  {\AuthorFont
    SOBRENOME, Nome do Autor$^{1}$; SOBRENOME, Nome do Coautor$^{1}$\\[0.2cm]
    \normalsize $^{1}$Curso / Departamento – NOME DA INSTITUIÇÃO\\[0.2cm]
    Orientador(a): SOBRENOME, Nome do(a) Orientador(a)
  }
\end{minipage}

\vspace{0.5cm}
\hrule height 2pt
\vspace{0.5cm}


\end{document}
