\documentclass[10pt]{beamer}

\usetheme{Madrid}
\usecolortheme{seahorse}

\usepackage{graphicx}
\usepackage{adjustbox}
\usepackage{amsmath}
\usepackage{bm}
\usepackage{xcolor}

% Cores personalizadas
\definecolor{academicblue}{RGB}{0,51,102}
\definecolor{lightblue}{RGB}{230,240,250}

\setbeamercolor{title}{fg=academicblue}
\setbeamercolor{frametitle}{fg=academicblue}

\title[PVQD]{Projeção Variacional da Dinâmica Quântica (PVQD)}
\author{Lucas F. Brayner \\ Luiz Felipe Sá \\ Victor S. Dutra}
\institute{Ciência da Computação -- UNINASSAU}
\date{2025}

% -------------------------------------------------------
\begin{document}

\begin{frame}
  \titlepage
\end{frame}

% -------------------------------------------------------
\begin{frame}{Motivação}
  \begin{itemize}
    \item Simular a evolução temporal de sistemas quânticos é computacionalmente caro.
    \item Trotterização exige circuitos profundos $\rightarrow$ difícil para hardware NISQ.
    \item O PVQD evita decompor $e^{-iHt}$ e usa otimização variacional.
    \item Bom cenário de teste: modelo de Ising com poucos qubits.
  \end{itemize}
\end{frame}

% -------------------------------------------------------
\begin{frame}{O que é o PVQD?}
  \begin{itemize}
    \item Método híbrido quântico--clássico.
    \item Aproxima a dinâmica pela equação de Schrödinger:
    \[
      i \frac{d}{dt}\lvert \psi(t)\rangle = H \lvert \psi(t)\rangle .
    \]
    \item Restringimos o estado a um ansatz parametrizado:
    \[
      \lvert \psi(\boldsymbol{\theta}(t))\rangle.
    \]
    \item A evolução é obtida atualizando os parâmetros em cada passo de tempo.
  \end{itemize}
\end{frame}

% -------------------------------------------------------
\begin{frame}{Equação variacional do PVQD}
\[
  \min_{\dot{\boldsymbol{\theta}}}
  \left\|
    \frac{d}{dt} \lvert \psi(\boldsymbol{\theta})\rangle
    - H \lvert \psi(\boldsymbol{\theta})\rangle
  \right\|^2
\]

\begin{itemize}
  \item Isso leva a um sistema linear:
    \[
      M \, \dot{\boldsymbol{\theta}} = V,
    \]
    onde $M$ é a matriz variacional e $V$ o vetor de forças.
  \item O algoritmo resolve esse sistema a cada passo para atualizar os parâmetros.
\end{itemize}
\end{frame}

% -------------------------------------------------------
\begin{frame}{Fluxo geral do algoritmo PVQD}
\begin{enumerate}
    \item Definir Hamiltoniano $H$ (modelo físico).
    \item Escolher ansatz variacional $\lvert \psi(\boldsymbol{\theta})\rangle$.
    \item Medir grandezas necessárias para $M$ e $V$.
    \item Resolver $M \dot{\boldsymbol{\theta}} = V$.
    \item Atualizar parâmetros e avançar no tempo.
    \item Medir observáveis: energia, correlação, fidelidade.
\end{enumerate}

\begin{center}
\includegraphics[width=0.65\linewidth]{images/pvqd_geometry_flow.png}
\end{center}
\end{frame}

% -------------------------------------------------------
\begin{frame}{Modelo de Ising estudado}
Hamiltoniano:
\[
H = J \sum_i Z_i Z_{i+1} + h_x \sum_i X_i
\]

\begin{itemize}
    \item Parâmetros escolhidos: $J = 0{,}5$ e $h_x = 0{,}1$ (configuração padrão).
    \item Sistema de 2 qubits, com acoplamentos $ZZ$ e campo transversal em $X$.
    \item Simulação em modo \emph{statevector} usando Qiskit.
\end{itemize}
\end{frame}

% -------------------------------------------------------
\begin{frame}{Ansatz utilizado}
\begin{itemize}
  \item Circuito do tipo \texttt{EfficientSU2} com:
    \begin{itemize}
      \item Rotações locais em cada qubit.
      \item Camada de emaranhamento linear.
      \item Apenas 1 repetição (reps = 1).
    \end{itemize}
  \item Total de 8 parâmetros para 2 qubits.
\end{itemize}

\begin{center}
\includegraphics[width=0.55\linewidth]{images/ansatz_layer.png}
\end{center}
\end{frame}

% -------------------------------------------------------
% ARQUITETURA + PRINTS DE CÓDIGO
% -------------------------------------------------------

\begin{frame}{Arquitetura do código PVQD}
  \begin{itemize}
    \item \textbf{config.py}: define parâmetros físicos e numéricos da simulação.
    \item \textbf{hamiltonian.py}: constrói o Hamiltoniano de Ising e o observável $ZZ$.
    \item \textbf{circuit.py}: cria o ansatz variacional (\texttt{EfficientSU2}).
    \item \textbf{pvqd\_solver.py}: configura o algoritmo PVQD da Qiskit.
    \item \textbf{results.py}: imprime tempos, fidelidades, energia e correlações.
  \end{itemize}
\end{frame}

\begin{frame}{Parâmetros da simulação: \texttt{config.py}}
  \begin{center}
    \adjustbox{max width=0.95\linewidth,max height=0.85\textheight,center}{\includegraphics{images/config.png}}
  \end{center}
\end{frame}

\begin{frame}{Modelo físico: \texttt{hamiltonian.py}}
  \begin{center}
    \adjustbox{max width=0.95\linewidth,max height=0.85\textheight,center}{\includegraphics{images/hamiltonian.png}}
  \end{center}
\end{frame}

\begin{frame}{Ansatz variacional: \texttt{circuit.py}}
  \begin{center}
    \adjustbox{max width=0.95\linewidth,max height=0.85\textheight,center}{\includegraphics{images/circuit.png}}
  \end{center}
\end{frame}

\begin{frame}{Algoritmo PVQD: \texttt{pvqd\_solver.py}}
  \begin{center}
    \adjustbox{max width=0.95\linewidth,max height=0.85\textheight,center}{\includegraphics{images/pvqd_solver.png}}
  \end{center}
\end{frame}

\begin{frame}{Pós-processamento: \texttt{results.py}}
  \begin{center}
    \adjustbox{max width=0.95\linewidth,max height=0.85\textheight,center}{\includegraphics{images/results.png}}
  \end{center}
\end{frame}

\begin{frame}{Fluxo principal: \texttt{main.py}}
  \begin{center}
    \adjustbox{max width=0.95\linewidth,max height=0.85\textheight,center}{\includegraphics{images/main.png}}
  \end{center}
\end{frame}

% -------------------------------------------------------
\begin{frame}{Parâmetros da simulação}
\begin{itemize}
  \item Qubits: 2
  \item Passos de tempo: 10
  \item Tempo total: $T = 1{,}0$
  \item Parâmetros do ansatz: 8
  \item Otimizador clássico: até 20 avaliações por passo (\texttt{L\_BFGS\_B})
\end{itemize}
\end{frame}

% -------------------------------------------------------
\begin{frame}{Resultado: fidelidade}
\begin{center}
\begin{tabular}{c c}
\hline
$t$ & Fidelidade \\
\hline
0.0 & 1.0000 \\
0.5 & 0.9990 \\
1.0 & 0.9980 \\
\hline
\end{tabular}

\vspace{0.3cm}

Fidelidade mínima observada: $\mathcal{F}_{\min} \approx 0{,}998$.
\end{center}
\end{frame}

% -------------------------------------------------------
\begin{frame}{Resultado: energia e correlação}
\begin{center}
\begin{tabular}{c c c}
\hline
$t$ & $\langle H\rangle$ & $\langle ZZ\rangle$ \\
\hline
0.0 & 0.500000 & 1.000000 \\
0.5 & 0.503872 & 0.999598 \\
1.0 & 0.507375 & 0.998441 \\
\hline
\end{tabular}
\end{center}

\begin{itemize}
  \item Energia varia suavemente ao longo do tempo.
  \item Correlação $\langle ZZ\rangle$ próxima de 1 indica forte correlação de spins.
\end{itemize}
\end{frame}

% -------------------------------------------------------
\begin{frame}{Conclusões}
\begin{itemize}
  \item PVQD reproduziu bem a dinâmica do modelo de Ising em 2 qubits.
  \item Um ansatz raso com 8 parâmetros foi suficiente para alta fidelidade.
  \item Método promissor para simulações em hardware NISQ, com circuitos mais curtos que Trotterização.
  \item Próximo passo: comparação sistemática com hardware IBM Quantum.
\end{itemize}
\end{frame}

% -------------------------------------------------------
\begin{frame}{Referências}
{\footnotesize
1. Qiskit Algorithms. \emph{Projected Variational Quantum Dynamics Tutorial}. \\
2. Barison, S.; Vicentini, F.; Carleo, G. \emph{An efficient quantum algorithm for the time evolution of parameterized circuits}. Quantum 5, 512 (2021). \\
3. IBM Quantum. \emph{PVQD API Documentation}. \\
4. Barison, S.; Vicentini, F.; Carleo, G. arXiv:2101.04579 (2021).
}
\end{frame}

% -------------------------------------------------------
\end{document}
