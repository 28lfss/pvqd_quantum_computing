\documentclass[10pt]{beamer}

% Tema limpo e profissional
\usetheme{Madrid}
\usecolortheme{seahorse}

\usepackage{graphicx}
\usepackage{amsmath}
\usepackage{bm}
\usepackage{listings}
\usepackage{xcolor}

% Cores personalizadas (baseado no banner)
\definecolor{academicblue}{RGB}{0,51,102}
\definecolor{lightblue}{RGB}{230,240,250}

\setbeamercolor{title}{fg=academicblue}
\setbeamercolor{frametitle}{fg=academicblue}

\lstset{
    language=Python,
    basicstyle=\ttfamily\footnotesize,
    keywordstyle=\color{academicblue}\bfseries,
    commentstyle=\color{gray}\itshape,
    stringstyle=\color{academicblue},
    showstringspaces=false,
    breaklines=true
}

\title[PVQD]{Projeção Variacional da Dinâmica Quântica (PVQD)}
\author{Lucas F. Brayner \\ Luiz Felipe Sá \\ Victor S. Dutra}
\institute{Ciência da Computação – UNINASSAU}
\date{2025}

% -------------------------------------------------------
\begin{document}

\begin{frame}
  \titlepage
\end{frame}

% -------------------------------------------------------
\begin{frame}{Motivação}
  \begin{itemize}
    \item Simular a evolução temporal de sistemas quânticos é computacionalmente caro.
    \item Trotterização exige circuitos profundos → difícil para hardware NISQ.
    \item O PVQD evita decompor $e^{-iHt}$ e usa otimização variacional.
    \item Ótimo para sistemas pequenos, como o modelo de Ising.
  \end{itemize}
\end{frame}

% -------------------------------------------------------
\begin{frame}{O que é o PVQD?}
  \begin{itemize}
    \item Método híbrido quântico–clássico.
    \item Aproxima a dinâmica pela projeção da equação de Schrödinger:
    \[
      i \frac{d}{dt}|\psi(t)\rangle = H|\psi(t)\rangle.
    \]
    \item O estado é restrito a um ansatz parametrizado:
    \[
      |\psi(\theta(t))\rangle.
    \]
    \item A evolução é obtida ajustando $\dot{\theta}$ em cada passo.
  \end{itemize}
\end{frame}

% -------------------------------------------------------
\begin{frame}{Equação Variacional}
\[
  \min_{\dot{\bm{\theta}}}
  \Big\|
    \frac{d}{dt}|\psi(\bm{\theta})\rangle
    - H|\psi(\bm{\theta})\rangle
  \Big\|^2
\]
\begin{itemize}
  \item Isso gera um sistema linear:
  \[
    M \cdot \dot{\bm{\theta}} = V,
  \]
  onde $M$ é a matriz variacional e $V$ é o vetor de forças.
  \item Resolver este sistema fornece os incrementos dos parâmetros.
\end{itemize}
\end{frame}

% -------------------------------------------------------
\begin{frame}{Fluxo geral do algoritmo PVQD}
\begin{enumerate}
    \item Definir Hamiltoniano $H$.
    \item Definir ansatz variacional $|\psi(\theta)\rangle$.
    \item Medir $M$ e $V$.
    \item Resolver $M\dot{\theta} = V$.
    \item Atualizar parâmetros e avançar no tempo.
    \item Medir observáveis (energia, correlação, fidelidade).
\end{enumerate}

\begin{center}
\includegraphics[width=0.65\linewidth]{images/pvqd_geometry_flow.png}
\end{center}
\end{frame}

% -------------------------------------------------------
\begin{frame}{Modelo de Ising estudado}
Hamiltoniano:
\[
H = J \sum_i Z_i Z_{i+1} + h_x \sum_i X_i
\]

\begin{itemize}
    \item Utilizamos $J = 1$, $h_x = 1$.
    \item Sistema de 2 qubits.
    \item Ideal para testar PVQD devido à forte correlação de spins.
\end{itemize}

\end{frame}

% -------------------------------------------------------
\begin{frame}{Ansatz utilizado}
\begin{itemize}
  \item Duas camadas com rotações locais + acoplamento $ZZ(\theta)$.
  \item Poucos parâmetros (8 no total).
\end{itemize}

\begin{center}
\includegraphics[width=0.55\linewidth]{images/ansatz_layer.png}
\end{center}

\end{frame}

% -------------------------------------------------------
\begin{frame}[fragile]{Código-base da simulação}

\begin{lstlisting}
config = PVQDConfig()
h = create_ising_hamiltonian(config)
ans = create_ansatz(config)
solver = PVQDSolver(config)

result = solver.solve(
    hamiltonian=h,
    aux_operators=[h]
)

ResultsProcessor(solver.estimator)\
    .display_results(result, h)
\end{lstlisting}

\end{frame}

% -------------------------------------------------------
\begin{frame}{Parâmetros da simulação}
\begin{itemize}
  \item Qubits: 2
  \item Passos de tempo: 10
  \item Tempo total: $T = 1.0$
  \item Parâmetros do ansatz: 8
  \item Otimizador: 20 iterações por passo
\end{itemize}
\end{frame}

% -------------------------------------------------------
\begin{frame}{Resultado: Fidelidade}
\begin{center}
\begin{tabular}{c c}
\hline
$t$ & Fidelidade \\
\hline
0.0 & 1.0000 \\
0.5 & 0.9990 \\
1.0 & 0.9980 \\
\hline
\end{tabular}

\vspace{0.3cm}

Alta fidelidade $\Rightarrow$ trajetória temporal bem aproximada.
\end{center}
\end{frame}

% -------------------------------------------------------
\begin{frame}{Resultado: Energia e correlação}
\begin{center}
\begin{tabular}{c c c}
\hline
$t$ & $\langle H\rangle$ & $\langle ZZ\rangle$ \\
\hline
0.0 & 0.500000 & 1.000000 \\
0.5 & 0.503872 & 0.999598 \\
1.0 & 0.507375 & 0.998441 \\
\hline
\end{tabular}
\end{center}

\begin{itemize}
  \item Energia suave → dinâmica consistente.
  \item Correlação próxima de 1 → forte alinhamento de spins.
\end{itemize}
\end{frame}

% -------------------------------------------------------
\begin{frame}{Conclusões}
\begin{itemize}
  \item O PVQD aproximou bem a evolução dinâmica do modelo de Ising.
  \item Ansätze rasos podem atingir fidelidades altas.
  \item Método promissor para simulações no regime NISQ.
  \item Evita circuitos longos baseados em Trotterização.
\end{itemize}
\end{frame}

% -------------------------------------------------------
\begin{frame}{Referências}
{\footnotesize
1. Qiskit PVQD Tutorial.  
2. Barison, Vicentini, Carleo (2021) – Quantum Journal.  
3. IBM Quantum PVQD API Documentation.  
4. Barison et al., arXiv:2101.04579.
}
\end{frame}

% -------------------------------------------------------
\end{document}
