\documentclass{beamer}

\usetheme{Madrid}

\usepackage[utf8]{inputenc}
\usepackage[T1]{fontenc}
\usepackage[brazil]{babel}
\usepackage{graphicx}
\usepackage{adjustbox}
\usepackage{amsmath}

\title[PVQD]{Projected Variational Quantum Dynamics}
\subtitle{Fundamentos Teóricos e Implementação}
\author{}
\institute{}
\date{}

\begin{document}

%------------------------------------------
\begin{frame}
  \titlepage
\end{frame}
%------------------------------------------

%------------------------------------------
\section{Motivação}
%------------------------------------------

\begin{frame}{Por que estudar dinâmica quântica?}
  \begin{itemize}
    \item A evolução temporal de sistemas quânticos é relevante para:
    \begin{itemize}
      \item química quântica,
      \item materiais,
      \item informação quântica,
      \item simulação de modelos físicos.
    \end{itemize}
    \item Computar essa evolução é difícil devido ao crescimento exponencial do espaço de estados.
    \item Métodos tradicionais exigem recursos inacessíveis para sistemas grandes.
  \end{itemize}
\end{frame}

\begin{frame}{Limitações dos dispositivos NISQ}
  \begin{itemize}
    \item Hardware atual possui:
    \begin{itemize}
      \item ruído,
      \item decoerência,
      \item número limitado de qubits,
      \item profundidade de circuito restrita.
    \end{itemize}
    \item Precisamos de algoritmos:
    \begin{itemize}
      \item rasos,
      \item robustos ao ruído,
      \item variacionais.
    \end{itemize}
  \end{itemize}
\end{frame}

%------------------------------------------
\section{Ideia geral do PVQD}
%------------------------------------------

\begin{frame}{O que é o PVQD?}
  \begin{itemize}
    \item Método variacional para aproximar a dinâmica de sistemas quânticos.
    \item Baseado na projeção da evolução física real em um subespaço acessível.
    \item Usa um ansatz parametrizado para representar o estado quântico.
    \item Atualiza parâmetros ao longo do tempo para seguir a dinâmica.
  \end{itemize}
\end{frame}

\begin{frame}{Ideia conceitual}
  \begin{itemize}
    \item A evolução real vive em todo o espaço de Hilbert.
    \item O ansatz vive em um subespaço parametrizado.
    \item PVQD: \textbf{projetar} a evolução ideal dentro do espaço variacional.
    \item Mantém a trajetória física enquanto respeita limitações do hardware.
  \end{itemize}
\end{frame}

\begin{frame}{Por que projeção variacional funciona?}
  \begin{itemize}
    \item Baseia-se no princípio de McLachlan:
    \begin{itemize}
      \item a melhor aproximação é aquela que minimiza a distância entre a evolução real e a variacional.
    \end{itemize}
    \item Permite capturar dinâmicas relevantes com poucos parâmetros.
    \item Elimina necessidade de implementar operadores de evolução profundos.
  \end{itemize}
\end{frame}

%------------------------------------------
\section{Formulação conceitual}
%------------------------------------------

\begin{frame}{Componentes principais do método}
  \begin{itemize}
    \item \textbf{Ansatz variacional}: família de estados acessíveis.
    \item \textbf{Derivadas variacionais}: como o estado muda com os parâmetros.
    \item \textbf{Métrica geométrica}: descreve a estrutura do espaço variacional.
    \item \textbf{Projeção da dinâmica}: aproximação da evolução temporal real.
    \item \textbf{Atualização dos parâmetros}: trajetória variacional ao longo do tempo.
  \end{itemize}
\end{frame}

\begin{frame}{Interpretação intuitiva}
  \begin{itemize}
    \item O estado evolui como um ponto movendo-se em uma superfície curva.
    \item O ansatz define uma sub-superfície mais simples.
    \item PVQD encontra a melhor trajetória dentro dessa sub-superfície.
    \item O método busca preservar:
    \begin{itemize}
      \item energia,
      \item coerência,
      \item estrutura geométrica.
    \end{itemize}
  \end{itemize}
\end{frame}

%------------------------------------------
\section{Arquitetura do Código}
%------------------------------------------

\begin{frame}{Estrutura do projeto}
  \begin{itemize}
    \item \texttt{config.py}: parâmetros da simulação.
    \item \texttt{hamiltonian.py}: construção do Hamiltoniano.
    \item \texttt{circuit.py}: definição do ansatz.
    \item \texttt{pvqd\_solver.py}: implementação do método PVQD.
    \item \texttt{main.py}: execução da dinâmica.
    \item \texttt{results.py}: análise de resultados.
  \end{itemize}
\end{frame}

\begin{frame}{config.py}
  \begin{columns}
    \column{0.5\textwidth}
    \begin{itemize}
      \item Define parâmetros da simulação usando \texttt{@dataclass}
      \item Parâmetros do Hamiltoniano: força de interação (J) e campo transverso (h\_x)
      \item Configurações de simulação: número de qubits, passos temporais, tempo total
      \item Parâmetros do ansatz: repetições e tipo de emaranhamento
      \item Configurações do otimizador: máximo de avaliações da função
    \end{itemize}
    
    \column{0.5\textwidth}
    \begin{center}
      \adjustbox{max width=0.48\textwidth,max height=0.75\textheight,center}{\includegraphics{code_images/config.png}}
    \end{center}
  \end{columns}
\end{frame}

\begin{frame}{hamiltonian.py}
  \begin{columns}
    \column{0.5\textwidth}
    \begin{itemize}
      \item Constrói o Hamiltoniano de Ising 1D: $H = J \sum_i Z_i Z_{i+1} + h_x \sum_i X_i$
      \item Usa \texttt{SparsePauliOp} do Qiskit para representação eficiente
      \item Implementa acoplamentos ZZ entre qubits vizinhos
      \item Adiciona termos de campo transverso (X) em cada qubit
      \item Define observável auxiliar para análise de resultados
    \end{itemize}
    
    \column{0.5\textwidth}
    \begin{center}
      \adjustbox{max width=0.48\textwidth,max height=0.75\textheight,center}{\includegraphics{code_images/hamiltonian.png}}
    \end{center}
  \end{columns}
\end{frame}

\begin{frame}{circuit.py}
  \begin{columns}
    \column{0.5\textwidth}
    \begin{itemize}
      \item Define o ansatz variacional usando \texttt{efficient\_su2} do Qiskit
      \item Ansatz parametrizado com número configurável de repetições
      \item Suporta diferentes padrões de emaranhamento (linear, circular, etc.)
      \item Inicializa parâmetros do ansatz (zero por padrão)
      \item Estrutura modular permite fácil troca de ansatz
    \end{itemize}
    
    \column{0.5\textwidth}
    \begin{center}
      \adjustbox{max width=0.48\textwidth,max height=0.75\textheight,center}{\includegraphics{code_images/circuit.png}}
    \end{center}
  \end{columns}
\end{frame}

\begin{frame}{pvqd\_solver.py}
  \begin{columns}
    \column{0.5\textwidth}
    \begin{itemize}
      \item Classe principal que encapsula o solver PVQD
      \item Configura primitivas (sampler, estimator, fidelity)
      \item Inicializa ansatz e parâmetros iniciais
      \item Usa otimizador L-BFGS-B para minimização variacional
      \item Resolve problema de evolução temporal via \texttt{PVQD.evolve()}
    \end{itemize}
    
    \column{0.5\textwidth}
    \begin{center}
      \adjustbox{max width=0.48\textwidth,max height=0.75\textheight,center}{\includegraphics{code_images/pvqd_solver.png}}
    \end{center}
  \end{columns}
\end{frame}

\begin{frame}{main.py}
  \begin{columns}
    \column{0.5\textwidth}
    \begin{itemize}
      \item Orquestra a execução completa da simulação
      \item Cria configuração, Hamiltoniano e observáveis
      \item Instancia o solver PVQD e executa a evolução temporal
      \item Processa e exibe resultados (energia, observáveis, fidelidades)
      \item Suporta integração opcional com backends IBM
    \end{itemize}
    
    \column{0.5\textwidth}
    \begin{center}
      \adjustbox{max width=0.48\textwidth,max height=0.75\textheight,center}{\includegraphics{code_images/main.png}}
    \end{center}
  \end{columns}
\end{frame}

\begin{frame}{results.py}
  \begin{columns}
    \column{0.5\textwidth}
    \begin{itemize}
      \item Processa e exibe resultados da simulação PVQD
      \item Mostra parâmetros, fidelidades e observáveis por timestep
      \item Visualiza circuito final evoluído
      \item Recomputa valores esperados localmente para validação
      \item Facilita análise comparativa entre timesteps
    \end{itemize}
    
    \column{0.5\textwidth}
    \begin{center}
      \adjustbox{max width=0.48\textwidth,max height=0.75\textheight,center}{\includegraphics{code_images/results.png}}
    \end{center}
  \end{columns}
\end{frame}

\begin{frame}{Outros módulos}
  \begin{columns}
    \column{0.5\textwidth}
    \textbf{primitives.py}
    \begin{itemize}
      \item Configura primitivas do Qiskit
      \item StatevectorSampler e StatevectorEstimator
      \item ComputeUncompute para cálculo de fidelidade
    \end{itemize}
    \vspace{0.2cm}
    \begin{center}
      \adjustbox{max width=0.48\textwidth,max height=0.35\textheight,center}{\includegraphics{code_images/primitives.png}}
    \end{center}
    
    \column{0.5\textwidth}
    \textbf{ibm\_backend.py}
    \begin{itemize}
      \item Integração com backends IBM Quantum
      \item Transpilação para gates nativos
      \item Avaliação em hardware real (opcional)
    \end{itemize}
    \vspace{0.2cm}
    \begin{center}
      \adjustbox{max width=0.48\textwidth,max height=0.35\textheight,center}{\includegraphics{code_images/ibm_backend.png}}
    \end{center}
  \end{columns}
\end{frame}

%------------------------------------------
\section{Conclusões}
%------------------------------------------

\begin{frame}{Conclusões}
  \begin{itemize}
    \item PVQD é um método eficiente para simular dinâmica em hardware NISQ.
    \item Baseia-se em princípios variacionais e projeções geométricas.
    \item Evita operadores profundos e é robusto ao ruído.
    \item Implementação modular facilita testes em simulador e hardware real.
  \end{itemize}
\end{frame}

\begin{frame}
  \centering
  \Large Obrigado!
\end{frame}

\end{document}
