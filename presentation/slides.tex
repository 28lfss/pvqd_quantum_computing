\documentclass{beamer}

\usetheme{Madrid}

\usepackage[utf8]{inputenc}
\usepackage[T1]{fontenc}
\usepackage[brazil]{babel}
\usepackage{amsmath, amssymb}
\usepackage{physics}
\usepackage{graphicx}

\title[PVQD]{Projected Variational Quantum Dynamics (PVQD)}
\subtitle{Conceito, Formulação e Implementação em Código}
\author{}
\institute{}
\date{}

\begin{document}

%------------------------------------------
\begin{frame}
  \titlepage
\end{frame}
%------------------------------------------

%------------------------------------------
\section{Motivação}
%------------------------------------------

\begin{frame}{Motivação}
  \begin{itemize}
    \item Simular evolução temporal quântica é computacionalmente difícil.
    \item Dispositivos NISQ não suportam circuitos profundos.
    \item É necessário um método variacional eficiente.
    \item PVQD projeta a evolução real dentro de um subespaço variacional acessível.
  \end{itemize}
\end{frame}

%------------------------------------------
\section{PVQD — Ideia Geral}
%------------------------------------------

\begin{frame}{Evolução Temporal}
  \[
    \ket{\psi(t+\Delta t)} = e^{-iH\Delta t}\ket{\psi(t)}
  \]
  \vspace{0.5cm}
  \begin{itemize}
    \item Evolução real é difícil de implementar diretamente.
    \item PVQD aproxima essa evolução usando um ansatz parametrizado.
  \end{itemize}
\end{frame}

\begin{frame}{Estado Variacional}
  \[
    \ket{\psi(\theta(t))}
  \]
  \begin{itemize}
    \item O ansatz determina os graus de liberdade acessíveis.
    \item A cada passo, ajustamos \(\theta\) para seguir a dinâmica real.
  \end{itemize}
\end{frame}

%------------------------------------------
\section{Formulação}
%------------------------------------------

\begin{frame}{Equações do PVQD}
  \[
    M \dot{\theta} = C
  \]
  \begin{itemize}
    \item Sistema linear que define a atualização dos parâmetros.
    \item Obtido pela projeção variacional da evolução temporal.
  \end{itemize}
\end{frame}

\begin{frame}{Componentes}
  \[
    M_{ij} = \Re(\braket{\partial_i\psi | \partial_j\psi}), \qquad
    C_i   = \Im(\matrixel{\partial_i\psi}{H}{\psi})
  \]
  \begin{itemize}
    \item \(M\): geometria do espaço variacional.
    \item \(C\): efeito do Hamiltoniano na direção variacional.
  \end{itemize}
\end{frame}

\begin{frame}{Atualização Variacional}
  \[
    \dot{\theta} = M^{-1}C
  \]
  \[
    \theta(t+\Delta t) = \theta(t) + \Delta t \,\dot{\theta}
  \]
  \begin{itemize}
    \item Evolução discreta dos parâmetros.
  \end{itemize}
\end{frame}

%------------------------------------------
\section{Arquitetura do Código}
%------------------------------------------

\begin{frame}{Arquitetura Geral}
  \begin{itemize}
    \item \texttt{config.py} — parâmetros da simulação.
    \item \texttt{hamiltonian.py} — definição do Hamiltoniano.
    \item \texttt{circuit.py} — ansatz variacional.
    \item \texttt{pvqd\_solver.py} — implementação do PVQD.
    \item \texttt{main.py} — pipeline completo.
    \item \texttt{results.py} — análises e gráficos.
  \end{itemize}
\end{frame}

\begin{frame}{config.py}
  \begin{center}
    \includegraphics[width=.9\linewidth]{code_images/config.png}
  \end{center}
\end{frame}

\begin{frame}{hamiltonian.py}
  \begin{center}
    \includegraphics[width=.9\linewidth]{code_images/hamiltonian.png}
  \end{center}
\end{frame}

\begin{frame}{circuit.py — Ansatz}
  \begin{center}
    \includegraphics[width=.9\linewidth]{code_images/circuit.png}
  \end{center}
\end{frame}

\begin{frame}{pvqd\_solver.py}
  \begin{center}
    \includegraphics[width=.9\linewidth]{code_images/pvqd_solver.png}
  \end{center}
\end{frame}

\begin{frame}{main.py}
  \begin{center}
    \includegraphics[width=.9\linewidth]{code_images/main.png}
  \end{center}
\end{frame}

\begin{frame}{results.py}
  \begin{center}
    \includegraphics[width=.9\linewidth]{code_images/results.png}
  \end{center}
\end{frame}

\begin{frame}{Outros módulos}
  \begin{center}
    \includegraphics[width=.8\linewidth]{code_images/primitives.png}\\[0.4cm]
    \includegraphics[width=.8\linewidth]{code_images/ibm_backend.png}
  \end{center}
\end{frame}

%------------------------------------------
\section{Conclusão}
%------------------------------------------

\begin{frame}{Conclusões}
  \begin{itemize}
    \item PVQD é adequado para simulações em hardware NISQ.
    \item Projeta a evolução temporal real no espaço variacional.
    \item Código implementa toda a cadeia: Hamiltoniano, ansatz, solver e análise.
  \end{itemize}
\end{frame}

\begin{frame}
  \centering
  \Large Obrigado!
\end{frame}

\end{document}
