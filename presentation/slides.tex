\documentclass{beamer}

% Theme
\usetheme{Madrid}
\usecolortheme{beaver}

% Packages
\usepackage[utf8]{inputenc}
\usepackage{amsmath}
\usepackage{physics}
\usepackage{graphicx}
\usepackage{braket}

\title[PVQD]{Projected Variational Quantum Dynamics (PVQD)}
\subtitle{Simulação Variacional da Dinâmica Quântica}
\author{Seu Nome}
\institute{UFPE}
\date{\today}

\begin{document}

% ---------------------------------------------------------
\begin{frame}
    \titlepage
\end{frame}

% ---------------------------------------------------------
\begin{frame}{Motivação}
    \begin{itemize}
        \item Simular a dinâmica de sistemas quânticos é computacionalmente caro.
        \item O espaço de Hilbert cresce exponencialmente: $2^n$ estados para $n$ qubits.
        \item Métodos variacionais permitem simular a evolução temporal em hardware NISQ.
        \item O PVQD surge como proposta eficiente e estável para evoluções dependentes do tempo.
    \end{itemize}
\end{frame}

% ---------------------------------------------------------
\begin{frame}{Ideia Central do PVQD}
    \begin{itemize}
        \item Aproximar a evolução da Equação de Schrödinger projetando-a sobre um espaço variacional.
        \item Utiliza um ansatz parametrizado $ \ket{\psi(\theta(t))} $.
        \item A dinâmica é obtida ponto a ponto, ajustando $\theta(t)$.
        \item Mais estável que VQS tradicional.
    \end{itemize}
\end{frame}

% ---------------------------------------------------------
\begin{frame}{Equação de Schrödinger}
    \[
        i \frac{d}{dt} \ket{\psi(t)} = H \ket{\psi(t)}
    \]
    \vspace{1em}
    \begin{itemize}
        \item Objetivo: aproximar a evolução sem calcular todo o vetor de estado.
        \item O PVQD transforma esse problema em equações diferenciais nos parâmetros do circuito.
    \end{itemize}
\end{frame}

% ---------------------------------------------------------
\begin{frame}{Ansatz Variacional}
    \begin{itemize}
        \item Ansatz: circuito quântico parametrizado.
        \item Define o subespaço acessível ao método.
        \item Exemplos:
        \begin{itemize}
            \item Hardware-Efficient Ansatz
            \item Ansatz inspirado no Hamiltoniano (QAOA-like)
        \end{itemize}
        \item Limitação: O método só evolui dentro do espaço gerado pelo ansatz.
    \end{itemize}
\end{frame}

% ---------------------------------------------------------
\begin{frame}{Projeção Variacional}
    \begin{itemize}
        \item O PVQD projeta $ \dot{\ket{\psi}} $ no espaço tangente do ansatz.
        \item Leva ao sistema linear:
    \end{itemize}
    \[
        M(\theta) \, \dot{\theta} = V(\theta)
    \]
    \begin{itemize}
        \item $M$: matriz de métricas (overlaps de derivadas do estado).
        \item $V$: termo que incorpora o Hamiltoniano.
        \item Resolver esse sistema fornece $\dot{\theta}$.
    \end{itemize}
\end{frame}

% ---------------------------------------------------------
\begin{frame}{Passo Temporal (Time-Stepping)}
    \begin{enumerate}
        \item Calcular $M(\theta)$ e $V(\theta)$ usando medidas quânticas.
        \item Resolver a equação linear para $\dot{\theta}$.
        \item Atualizar:  
        \[
            \theta_{t+\Delta t} = \theta_t + \dot{\theta}\Delta t
        \]
        \item Aplicar o circuito com novos parâmetros e repetir.
    \end{enumerate}
\end{frame}

% ---------------------------------------------------------
\begin{frame}{Exemplo de Hamiltoniano: Modelo de Ising}
    \[
        H = J\sum_i Z_i Z_{i+1} + h_x \sum_i X_i
    \]
    \begin{itemize}
        \item Hamiltoniano comum para testes.
        \item Mas o PVQD funciona com qualquer Hamiltoniano dependente ou não do tempo.
    \end{itemize}
\end{frame}

% ---------------------------------------------------------
\begin{frame}{Fluxo do Algoritmo PVQD}
    \begin{enumerate}
        \item Definir ansatz variacional.
        \item Medir expectativas necessárias.
        \item Resolver o sistema linear de McLachlan.
        \item Atualizar parâmetros.
        \item Evoluir o estado no tempo.
    \end{enumerate}

    % Espaço reservado para imagem
    \begin{center}
        \includegraphics[width=0.55\textwidth]{images/ansatz_layer.png}
    \end{center}
\end{frame}

% ---------------------------------------------------------
\begin{frame}{Resultados Típicos}
    \begin{itemize}
        \item Fidelidade entre evolução variacional e exata.
        \item Energia ao longo do tempo.
        \item Erro acumulado pequeno para $\Delta t$ moderado.
    \end{itemize}

    \begin{center}
        \includegraphics[width=0.55\textwidth]{images/ansatz_layer.png}
    \end{center}
\end{frame}

% ---------------------------------------------------------
\begin{frame}{Vantagens}
    \begin{itemize}
        \item Requer profundidade menor que Trotterização.
        \item Mais estável que VQS tradicional.
        \item Pode ser executado em hardware NISQ.
        \item Escalabilidade moderada.
    \end{itemize}
\end{frame}

% ---------------------------------------------------------
\begin{frame}{Limitações}
    \begin{itemize}
        \item Dependência forte da escolha do ansatz.
        \item Muitas medições para calcular $M$ e $V$.
        \item Possíveis barren plateaus.
        \item Erro cresce em longas evoluções temporais.
    \end{itemize}
\end{frame}

% ---------------------------------------------------------
\begin{frame}{Aplicações}
    \begin{itemize}
        \item Dinâmica de cadeias de spin.
        \item Materiais quânticos.
        \item Reações químicas.
        \item Simulação fora do equilíbrio.
    \end{itemize}
\end{frame}

% ---------------------------------------------------------
\begin{frame}{Conclusão}
    \begin{itemize}
        \item PVQD é um método promissor para simulação temporal quântica.
        \item Combina técnicas variacionais com projeção da dinâmica exata.
        \item Adequado para hardware NISQ.
    \end{itemize}
\end{frame}

% ---------------------------------------------------------
\begin{frame}{Referências}
    \footnotesize{
    Barison, S., Vicentini, F., \& Carleo, G. (2021). *Projected Variational Quantum Dynamics*. \\
    McLachlan, A. D. (1964). *A variational solution of the time-dependent Schrödinger equation*. \\
    Qiskit Textbook — Simulation Methods.
    }
\end{frame}

% ---------------------------------------------------------
\end{document}
